\documentclass[12pt, onecolumn]{article}
\usepackage[margin = 2cm]{geometry}
\usepackage{amssymb}
\usepackage{amsmath}

\title{\bf{UNIT-2}}
\author{Aaditya gupta}

\begin{document}
\maketitle

\section{Attenuation}
    \begin{equation}
        \alpha_{dB}L = 10 \log_{10} \frac{P_i}{P_o} \;\;\; L : \text{ length of fiber}
        \label{eq:attenuation}
    \end{equation}

\section{Linear Scattering Loss}
    \subsection{Rayleigh scattering}
        \begin{equation}
            \gamma_R = \displaystyle\frac{8\pi^3}{3\lambda^4}n^8p^2\beta_c K T_F
            \label{eq:rayleigh}
        \end{equation}

        where $ \gamma_R $ is Rayleigh scattering coefficient, $\lambda$ is optical wavelength, $n$ is refractive index of medium,
        $p$ is average photoelastic coefficient, $\beta_c$ is isothermal compressibility at a fictive temperature $T_F$ and $K$ is 
        Boltzmann constant

\section{Nonlinear Scattering Loss}
    \subsection{Stimulated Brillouin Scattering}
        \begin{equation}
            P_B = 4.4 \times 10^{-3} d^2\lambda^2\alpha_{dB}v \text{  watts}
            \label{eq: sbs}
        \end{equation} 
        where $d$ and $\lambda$ are fiber core diameter and operating wavelength, measured in
        micrometers. $v$ is source bandwidth.

    \subsection{Stimulated Raman Scattering}
        \begin{equation}
            P_R = 5.9 \times 10^{-2}d^2\lambda^2\alpha_{dB}
            \label{eq:srs}
        \end{equation}

\section{Fiber bend Loss}
    \begin{equation}
        R_c \simeq \displaystyle\frac{3n_1^2\lambda}{4\pi(n_1^2-n_2^2)^{1/2}}
        \label{eq:crit_radius}
    \end{equation}
    critical radius of curvature for single mode fiber
    \begin{equation}
        R_{cs} \simeq \displaystyle\frac{20\lambda}{(n_1 - n_2)^1/2} \left( 2.748 - 0.996 \displaystyle\frac{\lambda}{\lambda_c} \right)^{-3}
        \label{eq:crit_single_radius}
    \end{equation}

\section{Dispersion}
    \begin{equation}
        \beta = kn_1 [1-2\Delta(1-b)]^{1/2}
        \label{eq:beta}
    \end{equation}

    \begin{equation}
        B_T \leq \frac{1}{2\tau} \;\;\; \tau = \text{pulse duration due to dispersion}
        \label{eq:opt_bit_rate}
    \end{equation}
    
    \subsection{Chromatic dispersion : Material Dispersion}
        \begin{equation}
            \begin{aligned}
                \text{rms pulse broadening : }\sigma_m &= \displaystyle\frac{\sigma_\lambda L}{c} 
                        \left| \lambda \displaystyle\frac{d^2 n_1}{d\lambda^2} \right| \\
                \text{material dispersion parameter : }M &= \frac{\lambda}{c} \left| \displaystyle\frac{d^2 n_1}{d\lambda^2} \right| \\
                \sigma_m &= \sigma_\lambda L M \\
            \end{aligned}
            \label{eq:mat_dispersion}
        \end{equation}

    \subsection{Intermodal Dispersion}
        \begin{equation}
            \begin{aligned}
                \text{delay difference : } \delta T_s &= \frac{Ln_1\Delta}{c} = \frac{L(NA)^2}{2 n_1 c} \\
                \sigma_s &= \frac{L n_1 \Delta}{2 \sqrt 3 c} = \frac{L(NA)^2}{4\sqrt 3 n_1 c} \\
            \end{aligned} 
            \label{eq:intermodal_dispersion}
        \end{equation}

    \subsection{Overall fiber dispersion}
        \begin{equation}
            D_T(\lambda) = \displaystyle\frac{\lambda S_0}{4} \left[ 1 - \left(\displaystyle\frac{\lambda_0}{\lambda}\right)^2 \right]
            \label{eq:total_chromatic}
        \end{equation}
\end{document}
